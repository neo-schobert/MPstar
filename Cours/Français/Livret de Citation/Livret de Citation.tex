\documentclass[a4paper, 11pt, hidelinks]{article}
\usepackage{bookmark}
\usepackage[utf8]{inputenc} 
\usepackage[T1]{fontenc}
\usepackage{lmodern}
\usepackage{graphicx}
\usepackage[french]{babel}
\usepackage{geometry}
\usepackage{eucal}
\usepackage{caption}
\usepackage{float}
\usepackage{url}
\usepackage{amsmath}
\usepackage{amssymb}
\usepackage{color}
\usepackage{hyperref}
\usepackage{cancel}
\usepackage{tikz}
\usepackage{mathrsfs}  
\usepackage{esvect}
\usepackage[standard]{ntheorem}
\usepackage{romanbar}
\usepackage{titlesec}
\usepackage[final]{pdfpages}
\usepackage{xcolor}
\usepackage{xcolor, soul}



\geometry{hmargin=2cm,vmargin=1.5cm}

\tikzset{
  treenode/.style = {shape=rectangle, rounded corners,
                     draw, align=center,
                     top color=white, bottom color=blue!5},
  root/.style     = {treenode, font=\Large, bottom color=red!10},
  env/.style      = {treenode, font=\ttfamily\normalsize},
  dummy/.style    = {circle,draw}
}

\newcommand{\prp}{\large \textbf{Proposition :} \large}

\newcommand{\tm}{\large \textbf{Théoreme :} \large}

\newcommand{\ex}{\textcolor{green}{Exemple :} }

\newcommand{\dm}{\textcolor{red}{\textbf{Démo :} } }

\newcommand{\de}{\large \textbf{Définition} \large }

\newcommand{\rmq}{\textbf{Remarque :} }

\newcommand{\bs}{\bigskip}

\newcommand{\voca}{\textcolor{blue}{\textbf{Vocabulaire} } }

\newcommand{\lem}{\textcolor{red}{\textbf{Lemme :} } }

\newcommand{\rb}[1]{\Romanbar{#1}}

\newcommand{\cit}{\textcolor{blue}{\textbf{Citation : }}}


\definecolor{authorGray}{RGB}{20,20,20}
\definecolor{citationRed}{RGB}{255,110,110}
\definecolor{surlignage}{RGB}{255,249,151}

\newcommand{\citer}[3]{\bs \begin{center} \textcolor{authorGray}{#1 :} \textcolor{citationRed}{\og #2 \fg} \textcolor{authorGray}{(\underline{#3})} \end{center} \bs}


\newcommand{\trinom}[3]{\begin{pmatrix}
    #1 \\
    #2 \\
    #3
\end{pmatrix}}

\newcommand{\quadrinom}[4]{\begin{pmatrix}
    #1 \\
    #2 \\
    #3 \\
    #4 \\
\end{pmatrix}}

\newcommand{\pentanom}[5]{\begin{pmatrix}
    #1 \\
    #2 \\
    #3 \\
    #4 \\
    #5
\end{pmatrix}}

\newcommand{\hexanom}[6]{\begin{pmatrix}
    #1 \\
    #2 \\
    #3 \\
    #4 \\
    #5 \\
    #6 
\end{pmatrix}}

\newcommand{\serie}[2]{\displaystyle\sum_{#1 =0}^{+\infty} #2_{#1} }

\newcommand{\tend}{\underset{n \to + \infty}{\longrightarrow} }

\newcommand{\Lra}{\Leftrightarrow}

\newcommand{\lra}{\leftrightarrow}

\newcommand{\Ra}{\Rightarrow}

\newcommand{\ra}{\rightarrow}

\newcommand{\la}{\leftarrow}

\newcommand{\La}{\Leftarrow}

\newcommand{\dsum}[2]{\displaystyle\sum_{#1}^{#2} }

\newcommand{\dint}[2]{\displaystyle\int_{#1}^{#2} }

\newcommand{\ntend}{\underset{n \to + \infty}{\not \longrightarrow} }

\newenvironment{lmatrix}{$ \left|\begin{array}{l} }{\end{array}\right.$}

\newcommand{\img}[4]{\begin{figure}[!ht]
    \centering
    \includegraphics[scale=#1 ]{#2}
    \caption{#3}
    \label{#4}
    \end{figure} }    
\begin{document}

\newcommand{\grad}[1]{\vv{grad}#1}


\title{Livret de Citations}
\author{Schobert Néo}

\maketitle

\tableofcontents



\newpage


\section{Virgile - Les Géorgiques} 


\subsection{Livre 1}

\citer{Virgile}{et toi, alme Cérès, si, grâce à votre don, la terre a remplacé le gland de Chaonie par l'épi lourd, 
et versé dans la coupe de l'Achéloüs le jus des grappes par vous découvertes}{Les Géorgiques, Livre \rb{1} p $37-38$}


En parland de Octave, \citer{Virgile}{sensible comme moi aux misères des campagnards qui ne savent pas leur 
route}{Les Géorgiques, Livre \rb{1} p $40$}


\citer{Virgile}{Au printemps nouveau, quand fond la glace sur les monts chenus et que la glèbe amollie s'effrite
au doux Zéphyr, je veux dès lors voir le taureau commencer de gémir sous le poids de la charrue, et le soc
resplendir dans le sillon qu'il creuse}{Les Géorgiques, Livre \rb{1} p $41$}



\citer{Virgile}{La récolte ne comblera les voeux de l'avide laboureur que si elle a senti deux fois le soleil et deux 
fois les frimas: alors d'immenses moissons feront crouler ses greniers}{Les Géorgiques, Livre \rb{1} p $41$}


\citer{Virgile}{Ici les moissons viennent mieux; là, les raisins; ailleurs les fruits des arbres et les herbages verdoient d'eux-mêmes.}{Les Géorgiques, Livre \rb{1} p $41$}


\citer{Virgile}{Tes blés une fois coupés, tu laisseras la campagne se reposer pendant un an et, oisive, se durcir à l'abandon}{Les Géorgiques, Livre \rb{1} p $42$}


\citer{Virgile}{Mais pourant, grâce à l'alternance, le travail fourni par la terre est facile}{Les Géorgiques, Livre \rb{1} p $43$}


\citer{Virgile}{De plus, celui qui brise avec le hoyau les mottes inertes et qui fait passer sur elles les herses d'osier, fait
du bien aux guérets, et ce n'est pas pour rien que du haut de l'Olympe la blonde Cérès le regarde.}{Les Géorgiques, Livre \rb{1} p $43$}


\citer{Virgile}{Et cependant, en dépit de tout ce mal que les hommes et le boeufs se sont donné pour retourner la terre, ils ont encore 
à craindre l'oie vorace, les grues du Strymon, l'endive aux fibres amères et les méfaits de l'ombre.}{Les Géorgiques, Livre \rb{1} p $45$}


\citer{Virgile}{Le Père des dieux lui-même a voulu rendre la culture des champs difficile, et c'est lui qui le premier a fait un art 
de remuer la terre, en aiguisant par les soucis les c\oe urs des mortels et en ne souffrant pas que son empire s'engourdît dans une 
triste indolence.}{Les Géorgiques, Livre \rb{1} p $45$}


En parlant de l'avant Jupiter, \citer{Virgile}{les récoltes étaient mises en commun, et la terre produisait tout d'elle-même, 
librement, sans contrainte.}{Les Géorgiques, Livre \rb{1} p $45$}


En parlant de Jupiter, \citer{Virgile}{son but était, en exerçant le besoin, de créer peu à peu les différents arts, de faire 
chercher dans les sillons l'herbe du blé et jaillir du sein du caillou le feu qu'il recèle.}{Les Géorgiques, Livre \rb{1} p $45-46$}


\citer{Virgile}{Tous les obstacles furent vaincus par un travail acharné et par le besoin pressant en de dures circonstances.}{Les Géorgiques, Livre \rb{1} p $46-47$}


\citer{Virgile}{Si avec le hoyau tu ne fais pas une guerre assidue aux mauvaises herbes, si tu n'épouvantes à grand bruit
les oiseaux, si la serpe en main tu n'élagues l'ombrage qui recouvre ton champ, si tu n'appelles la pluie par tes v\oe ux, hélas ! 
tu en seras réduit à contempler le gros tas d'autrui et à secouer, pour soulager ta peine, le chêne dans les forêts.}{Les Géorgiques, Livre \rb{1} p $47$}


\citer{Virgile}{Tels sont les instruments que tu auras soin de te procurer longtemps d'avance, si tu veux mériter la gloire 
d'une campage divine.}{Les Géorgiques, Livre \rb{1} p $48$}


\citer{Virgile}{J'ai vu des semences, choisies à loisir et examinées avec beaucoup de soin, dégénérer pourtant, si chaque année 
on n'en triait à la main les plus belles: c'est une loi du destin que tout périclite et aille rétrogradant.}{Les Géorgiques, Livre \rb{1} p $50$}


\citer{Virgile}{Si d'aventure une pluie froide retient le cultivateur chez lui, il peut faire à loisir bien des ouvrages qu'il lui 
faudrait plus tard hâter par un ciel serein}{Les Géorgiques, Livre \rb{1} p $54$}


\citer{Virgile}{Oui, même aux jours de fête, il est des travaux auxquels les lois divines et humaines permettent de se livrer;}{Les Géorgiques, Livre \rb{1} p $54$}


\citer{Virgile}{Beaucoup de travaux nous sont rendus plus faciles par la fraîcheur de la nuit ou par la rosée dont, au lever 
du soleil, l'Aurore humecte les terres.}{Les Géorgiques, Livre \rb{1} p $55$}


En parlant de sa compagne, \citer{Virgile}{charmant par ses chansons l'ennui d'un long labeur}{Les Géorgiques, Livre \rb{1} p $56$}


\citer{Virgile}{Pendant les froids, les laboureurs jouissent d'ordinaire du fruit de leurs travaux, en donnant tour à tour
de gais festins entre eux.}{Les Géorgiques, Livre \rb{1} p $56$}


\citer{Virgile}{Ici-bas en effet le juste et l'injuste sont renversés, tant il y a de guerres par le monde, tant le crime
revêt d'aspects divers.}{Les Géorgiques, Livre \rb{1} p $68$}


\citer{Virgile}{La charrue ne reçoit plus l'honneur dont elle est digne; les guérets sont en friche, privés des laboureurs
entraînés dans les camps; et les faulx recourbées servent à forger une épée rigide.}{Les Géorgiques, Livre \rb{1} p $68$}



\subsection{Livre 2}

\citer{Virgile}{En effet les uns, sans y être contraints de la part des hommes, poussent d'eux-mêmes et couvrent au loin 
les plaines et les sinueuses vallées}{Les Géorgiques, Livre \rb{2} p $73$}


\citer{Virgile}{Mais d'autres naissent d'une semence qui s'est posée à terre}{Les Géorgiques, Livre \rb{2} p $74$}


\citer{Virgile}{D'autres habitants des forêts demandent qu'on courbe en arc leurs rejets et qu'on en plante les 
boutures dans leur propre terre.}{Les Géorgiques, Livre \rb{2} p $74$}


\citer{Virgile}{Au travail donc, ô cultivateurs! apprenez les procédés de culture propres à chaque espèce; adoucissez, en les cultivant,
les fruits sauvages; que vos terres ne restent pas en friche. Il y a plaisir à planter Bacchus sur l'Ismare et à vêtir d'oliviers
le grand Taburne.}{Les Géorgiques, Livre \rb{2} p $75$}


\citer{Virgile}{ils dépouilleront bientôt leur naturel sauvage et, cultivés avec soin, se plieront sans tarder à tous les 
artifices que l'on voudra.}{Les Géorgiques, Livre \rb{2} p $76$}


\citer{Virgile}{Au reste toute terre ne peut porter toute espèce de plantes.}{Les Géorgiques, Livre \rb{2} p $79$}


\citer{Virgile}{Ajoutez tant de villes incomparables, tant de travaux de construction, tant de places bâties par la main des 
hommes sur des rochers à pic, et ces fleuves baignant le pied d'antique murailles.}{Les Géorgiques, Livre \rb{2} p $82$}


\citer{Virgile}{Telle encore cette terre, d'où le laboureur irrité a fait disparaître une forêt, abattant des bocages longtemps
inutiles et arrachant jusqu'au bout de leurs racines les antiques demeures des oiseaux: eux ont abandonné leurs nids pour fuir
dans les airs, mais la plaine inculte a brillé sous le soc de la charrue.}{Les Géorgiques, Livre \rb{2} p $86$}


\citer{Virgile}{Il y a encore, parmi les soins dus aux vignes, un autre travail, et qui n'est jamais épuisé: il faut en effet
trois ou quatre fois l'an fendre tout le sol, et en briser éternellement les mottes avec le revers des bidents; il faut soulager
tout le vignoble de son feuillage.}{Les Géorgiques, Livre \rb{2} p $96$}


\citer{Virgile}{Le travail des laboureurs revient toujours en un cercle, et l'année en se déroulant le ramène avec elle sur ses 
traces.}{Les Géorgiques, Livre \rb{2} p $96$}


\citer{Virgile}{Fais l'éloge des vastes domaines, cultives-en un petit.}{Les Géorgiques, Livre \rb{2} p $97$}


\citer{Virgile}{Il faut encore couper dans la forêt les branches épineuses du houx, et sur ses rives le roseau fluvial; et il y a les
pénibles soins que demande la saulaie inculte.}{Les Géorgiques, Livre \rb{2} p $97$}


\citer{Virgile}{Maintenant les vignes sont liées; maintenant les arbustes laissent reposer la serpe; maintenant le vigneron au bout de 
ses rangées, chante la fin de ses peines. Pourtant il lui faut encore tourmenter la terre, la réduire en poussière, et, bientôt,
craindre Jupiter pour les raisins mûrs.}{Les Géorgiques, Livre \rb{2} p $97$}


\citer{Virgile}{Nourris donc le gras olivier agréable à la Paix.}{Les Géorgiques, Livre \rb{2} p $97$}


\citer{Virgile}{Et les hommes hésiteraient à planter des arbres et à y consacrer leurs soins !}{Les Géorgiques, Livre \rb{2} p $98$}


\citer{Virgile}{Il plaît de regarder le Cytore ondoyant sous le buis et les bois sacrés de l'arbre à poix de Naryce; il plaît
de voir des champs qui n'ont jamais été exposés aux hoyaux et à l'industrie de l'homme.}{Les Géorgiques, Livre \rb{2} p $98$}


\citer{Virgile}{Quel bienfait digne d'être autant célébré nous ont apporté les dons de Bacchus ?}{Les Géorgiques, Livre \rb{2} p $99$}


\citer{Virgile}{Ô trop fortunés, s'ils connaissaient leurs biens, les cultivateurs ! Eux qui, loin des discordes armées, voient la 
très juste terre leur verser de son sol une nourriture facile.}{Les Géorgiques, Livre \rb{2} p $99$}


\citer{Virgile}{du moins un repos assuré, une vie qui ne sait point les tromper, riche en ressources variées, du moins les loisirs en 
de vastes domaines, les grottes, les lacs d'eau vive, du moins les frais Tempé, les mugissements des b\oe ufs et les doux sommes sous 
l'arbre ne leur sont pas étrangers.}{Les Géorgiques, Livre \rb{2} p $100$}


\citer{Virgile}{Là où ils vivent sont les fourrés et les repaires des bêtes sauvages, une jeunesse dure aux travaux et habituée à peu,
le culte des dieux et le respect des pères; c'est chez eux qu'en quittant les terres la Justice laissa la trace de ses derniers pas.}{Les Géorgiques, Livre \rb{2} p $100$}


\citer{Virgile}{Mais fortuné aussi celui qui connaît les dieux champêtres, et Pan, et le vieux Silvain, et les Nymphes s\oe urs ! 
Celui-là, ni les faisceaux du peuple, ni la pourpre des rois ne l'ont fléchi, ni la discorde poussant des frères sans foi, ni le 
Dace descendant de l'Ister conjuré, ni les affaires de Rome, ni les royaumes destinés à périr; celui-là ne voit autour de lui
ni indigents à plaindre miséricordieusement, ni riches à envier. Les fruits que donnent les rameaux, ceux que donnent d'elles-mêmes
les bienveillantes campagnes, il les ceuille sans connaître ni les lois d'airain ni le forum insensé ni les archives du peuple.}{Les Géorgiques, Livre \rb{2} p $102-103$}


\citer{Virgile}{Le laboureur fend la terre de son areau incurvé: c'est de là que découle le labeur de l'année; c'est par là que découle
le labeur de l'année; c'est par là qu'il sustente sa patrie et ses petits-enfants, ses troupeaux de b\oe ufs et ses jeunes taureaux
qui l'ont bien mérité.}{Les Géorgiques, Livre \rb{2} p $103$}


\citer{Virgile}{Cependant ses enfants câlins suspendus à son cou se disputent ses baisers; sa chaste demeure observe la pudicité;
ses vaches laissent pendre leurs mamelles pleines de lait, et ses gros chevreaux, cornes contre cornes, luttent entre eux sur le
riant gazon.}{Les Géorgiques, Livre \rb{2} p $104$}


\citer{Virgile}{Lui aussi a ses jours de fête, où, allongé sur l'herbe, tandis qu'au milieu brûle un feu sacré et que ses compagnons
couronnent les cratères, il t'invoque, Lénéen, avec une libation, puis invite les gardiens du troupeau à lancer un rapide javelot sur 
la cible d'un orme et à dépouiller leurs corps rudes pour la palestre champêtre.}{Les Géorgiques, Livre \rb{2} p $104$}


\citer{Virgile}{Telle est la vie que menèrent jadis les vieux Sabins, telle fut celle de Rémus et de son frère. Ainsi grandit sans 
doute la vaillante Etrurie; ainsi Rome devint la merveille du monde et seule dans son enceinte renferma sept collines.}{Les Géorgiques, Livre \rb{2} p $104$}


\citer{Virgile}{et avant qu'une race impie se fût nourrie de la chair des taureaux égorgés, telle fut la vie que menait sur les terres
Saturne d'or: on n'avait point alors entendu encore souffler dans les clairons, ni sur les dures enclumes crépiter les épées.}{Les Géorgiques, Livre \rb{2} p $105$}



\subsection{Livre 3}


\citer{Virgile}{Les plus beaux jours de l'âge des malheureux mortels sont les premiers à fuir: à leur place viennent les maladies
et la triste vieillesse, puis les souffrances, et l'inclémence de la dure mort nous prend.}{Les Géorgiques, Livre \rb{3} p $114-115$}


\citer{Virgile}{Point de trêve, point de relâche ! Un nuage de poussière fauve s'élève; ils sont mouillés de l'écume et du souffle de 
ceux qui les suivent: tant l'amour de la gloire est grand, tant ils ont la victoire à c\oe ur !}{Les Géorgiques, Livre \rb{3} p $117$}


\citer{Virgile}{Quand au bout de quelques mois elles errent, chargées de leur fruit, qu'on ne les laisse point mener sous le joug des 
chariots lourds, ni franchir un chemin en sautant, ni s'enfuir au galop dans les prés, ni se jeter à la nage dans des eaux rapides.}{Les Géorgiques, Livre \rb{3} p $119$}


\citer{Virgile}{Ceux que tu veux former aux soins et aux besoins de la campagne, entraîne-les quand ils sont encore de petits veaux,
et engage-toi dans la voie du dressage, tandis que leur humeur est docile encore et leur jeune âge facile à plier.}{Les Géorgiques, Livre \rb{3} p $120$}


\citer{Virgile}{Mais le meilleur moyen d'affermir la vigueur, soit des b\oe ufs soit des chevaux, selon ce qu'on préfère, est d'écarter 
Vénus et les aiguillons de l'amour aveugle.}{Les Géorgiques, Livre \rb{3} p $122-123$}


\citer{Virgile}{Oui, toute la race sur terre et des hommes et des bêtes, ainsi que la race marine, les troupeaux, 
les oiseaux peints de mille couleurs, se ruent à ces furies et à ce feu: l'amour est le même pour tous.}{Les Géorgiques, Livre \rb{3} p $124$}


\citer{Virgile}{Que n'ose point un jeune homme, lorsque le dur amour fait circuler dans ses os son feu puissant ?}{Les Géorgiques, Livre \rb{3} p $125$}


\citer{Virgile}{C'est un travail; mais espérez-en de la gloire, courageux cultivateurs.}{Les Géorgiques, Livre \rb{3} p $127$}


\citer{Virgile}{Ces barbares mènent une vie tranquille et oisive dans des cavernes creusées profondément sous terre, entassant 
des rouvres et des ormes entiers pour les rouler sur leurs foyers et les livrer aux flammes.}{Les Géorgiques, Livre \rb{3} p $132$}


\citer{Virgile}{Il succombe, malheureux, oubliant la gloire et la prairie, le cheval vainqueur; il se détourne des fontaines, et,
du pied, frappe sans cesse la terre; ses oreilles baissées distillent une sueur incertaine, qui devient froide quand la mort
approche; sa peau est sèche, et, rugueuse, résiste à la main qui la touche.}{Les Géorgiques, Livre \rb{3} p $138$}


\citer{Virgile}{Mais voici que, fumant sous la dure charrue, le taureau s'affaisse et vomit à plein gosier un sang mêlé d'écume, et 
pousse de suprêmes gémissements. Le laboureur s'en va, tout triste, dételer l'autre b\oe uf affligé de la mort de son frère et laisse 
sa charrue enfoncée au milieu du sillon.}{Les Géorgiques, Livre \rb{3} p $139$}


\citer{Virgile}{Que leur servent leur labeur et leurs bienfaits ? que leur sert d'avoir retourné avec le soc de lourdes terres ?}{Les Géorgiques, Livre \rb{3} p $140$}


\citer{Virgile}{Alors donc les habitants du pays fendent à grande peine la terre avec les herses, enfouissent les semences avec leurs 
ongles mêmes, et gravissent les montagnes en traînant, le cou tendu, de gémissants chariots.}{Les Géorgiques, Livre \rb{3} p $140$}


\citer{Virgile}{L'air est funeste aux oiseaux eux-mêmes, et ils tombent, laissant la vie au haut des nues.}{Les Géorgiques, Livre \rb{3} p $141$}


\citer{Virgile}{En outre, peu importe qu'on change de pâturages; les remèdes cherchés sont nuisibles; les maîtres de l'art, Chiron,
fils de Philyre, et Mélampus, fils d'Amythaon, cèdent à la force du mal.}{Les Géorgiques, Livre \rb{3} p $141$}





\subsection{Livre 4}


\begin{itemize}
    \item 
\end{itemize}


\section{Simone Weil - La Condition ouvrière}






\end{document}