\documentclass[a4paper, 11pt, hidelinks]{article}
\usepackage{bookmark}
\usepackage[utf8]{inputenc} 
\usepackage[T1]{fontenc}
\usepackage{lmodern}
\usepackage{graphicx}
\usepackage[french]{babel}
\usepackage{geometry}
\usepackage{eucal}
\usepackage{caption}
\usepackage{float}
\usepackage{url}
\usepackage{amsmath}
\usepackage{amssymb}
\usepackage{color}
\usepackage{hyperref}
\usepackage{cancel}
\usepackage{tikz}
\usepackage{mathrsfs}  
\usepackage{esvect}
\usepackage[standard]{ntheorem}
\usepackage{romanbar}
\usepackage{titlesec}



\geometry{hmargin=2cm,vmargin=1.5cm}

\tikzset{
  treenode/.style = {shape=rectangle, rounded corners,
                     draw, align=center,
                     top color=white, bottom color=blue!5},
  root/.style     = {treenode, font=\Large, bottom color=red!10},
  env/.style      = {treenode, font=\ttfamily\normalsize},
  dummy/.style    = {circle,draw}
}

\newcommand{\prp}{\large \textbf{Proposition :} \large}

\newcommand{\tm}{\large \textbf{Théoreme :} \large}

\newcommand{\ex}{\textcolor{green}{Exemple :} }

\newcommand{\dm}{\textcolor{red}{\textbf{Démo :} } }

\newcommand{\de}{\large \textbf{Définition} \large }

\newcommand{\rmq}{\textbf{Remarque :} }

\newcommand{\bs}{\bigskip}

\newcommand{\voca}{\textcolor{blue}{\textbf{Vocabulaire} } }

\newcommand{\lem}{\textcolor{red}{\textbf{Lemme :} } }

\newcommand{\rb}[1]{\Romanbar{#1}}

\newcommand{\cit}{\textcolor{blue}{\textbf{Citation : }}}

\newcommand{\trinom}[3]{\begin{pmatrix}
    #1 \\
    #2 \\
    #3
\end{pmatrix}}

\newcommand{\quadrinom}[4]{\begin{pmatrix}
    #1 \\
    #2 \\
    #3 \\
    #4 \\
\end{pmatrix}}

\newcommand{\pentanom}[5]{\begin{pmatrix}
    #1 \\
    #2 \\
    #3 \\
    #4 \\
    #5
\end{pmatrix}}

\newcommand{\hexanom}[6]{\begin{pmatrix}
    #1 \\
    #2 \\
    #3 \\
    #4 \\
    #5 \\
    #6 
\end{pmatrix}}

\newcommand{\serie}[2]{\displaystyle\sum_{#1 =0}^{+\infty} #2_{#1} }

\newcommand{\tend}{\underset{n \to + \infty}{\longrightarrow} }

\newcommand{\Lra}{\Leftrightarrow}

\newcommand{\lra}{\leftrightarrow}

\newcommand{\Ra}{\Rightarrow}

\newcommand{\ra}{\rightarrow}

\newcommand{\la}{\leftarrow}

\newcommand{\La}{\Leftarrow}

\newcommand{\dsum}[2]{\displaystyle\sum_{#1}^{#2} }

\newcommand{\dint}[2]{\displaystyle\int_{#1}^{#2} }

\newcommand{\ntend}{\underset{n \to + \infty}{\not \longrightarrow} }

\newenvironment{lmatrix}{$ \left|\begin{array}{l} }{\end{array}\right.$}

\newcommand{\img}[4]{\begin{figure}[!ht]
    \centering
    \includegraphics[scale=#1 ]{#2}
    \caption{#3}
    \label{#4}
    \end{figure} }    
\begin{document}

\newcommand{\grad}[1]{\vv{grad}#1}


\title{Vocabulaire Virgile - Les Georgiques}
\author{Schobert Néo}

\maketitle

\tableofcontents


\newpage



\section{Livre 1 - Le labourage}

\begin{enumerate}
    \item Qu'est-ce que les \textcolor{red}{douze astres du monde} ? : Il s'agit des douze signes du Zodiaque : Bélier (Minerve), Taureau (Vénus), Gémeaux (Apollon),
          Cancer (Mercure), Lion (Jupiter), Vierge (Cérès), Balance (Vulcain), Scorpion (Mars), Sagitaire (Diane), Capricorne (Vesta), Verseau (Junon), Poisson (Neptune).
    \item Qu'est-ce que les \textcolor{red}{échalas} ? : Pieu servant à soutenir un pied de vigne ou autre arbuste.
    \item Qu'est-ce qu'une \textcolor{red}{éteule} ? : Chaume restant sur la terre après la moisson.
    \item Qu'est-ce qu'une \textcolor{red}{étoupe} ? : Excédent de filasse.
    \item Quelles sont les \textcolor{red}{constellations de l'automne} ? : Il s'agit de l'Arcture, des Chevreaux, de la Couronne, et du Centaure.
    \item Qu'est-ce que \textcolor{red}{l'Athos} ? : Il s'agit d'un mont de macédoine.
          \img{0.35}{Athos.png}{Mont Athos}{40}
    \item Qu'est-ce que le \textcolor{red}{Rhodope} ? : Il s'agit de montagnes de la Thrace.
          \img{0.4}{Rhodope.png}{Mont Rhodope}{41}
    \item Qu'est-ce que le \textcolor{red}{feu de Cyllène} ? : Il s'agit de Mercure, qui est né sur le mont Cyllène (auj. Ziria) en Arcadie.
          \img{0.3}{Cyllène.png}{Cyllène}{43}
    \item Pourquoi les \textcolor{red}{grenouilles chantent leur antique complainte} ? : Ovide conte que Latone, insulté par des paysans, obtint de Jupiter
          qu'ils fussent changés en grenouilles.
    \item Qu'est-ce que le \textcolor{red}{Caystre} ? : Il s'agit d'un fleuve de Lydie.
          \img{0.4}{Caystre.png}{Caystre}{46}
    \item Qui est \textcolor{red}{Phébé} ? : Phébé est une Titanides. Elle est la mère de Latone (Léto) et épouse de son frère Céos. Elle portait une couronne d'or.
    \item Qui est \textcolor{red}{Glaucus} ? : Glaucus, pêcheur d'Anthédon, en Béotie, ayant mangé une herbe magique, se précipita dans la mer, où il fut changé en dieu marin.
    \item Qui est \textcolor{red}{Panopée} ? : Panopée, l'une des cinquantes filles de Nérée et de Doris, celle dont il est question dans l'Illiade.
    \item Qui est \textcolor{red}{l'Inoen Mélicerte} ? : Mélicerte, fils d'Ino, fuyant avec sa mère son père Athamas, devenu fou furieux, s'élança avec
          elle dans la mer, et tous deux furent changés en divinités marines : Mélicerte sous le nom grec de Palémon et latin de Portumne; Ino, sous
          ceux de Leucothoé et de Matuta.
    \item Qui est \textcolor{red}{Tithon} ? : Prince troyen, fils de Laomédon et frère de Priam. Il fut enlevé par l'Aurore, éprise de lui, qui eut de lui
          deux fils, Memnon et Emathéon. Aurore l'adorait et sortait à l'aube, exténuée et pâle de sa couche. Elle obtint pour lui l'immortalité mais
          ayant omis de demander une éternelle jeunesse, Tithon vieillit, tomba en décrépitude, et fut finalement changé en cigale.
    \item Qu'est-ce que le \textcolor{red}{pampre} ? : Le pampre est une tige de vigne portant ses feuilles, ses vrilles et, souvent, ses grappes de raisin.
    \item Qu'est-ce que \textcolor{red}{l'Eridan} ? : Il s'agit du pô, le plus long fleuve d'Italie.
          \img{0.5}{Eridan.jpg}{Eridan}{47}
    \item Où est \textcolor{red}{l'Emathie} ? :
          \img{0.3}{Emathie.png}{Emathie}{48}
    \item Où est \textcolor{red}{l'Hémus} ? :
          \img{0.3}{Hémus.png}{Hémus}{49}
    \item Qui est \textcolor{red}{Vesta} ? : Vesta (Hestia), Déesse du feu, fille de Saturne (Cronos) et Cybèle (Rhéa) et soeur de Jupiter (Zeus), Neptune (Poséidon), Pluton (Hadès), Junon (Héra) et
          Cérès (Déméter) elle symbolise le foyer de la maison, bien que le culte romain ait pris plus d’importance.
    \item Qu'est-ce que les \textcolor{red}{parjures de la Troie Laomédontienne} ? La Troie Laomédontienne est la ville de Troie au temps où Laomédon y régnait.
          Ce dernier y a commis quelques parjures envers les Dieux. Pour la construction de la ville de Troie, il a promis à Apollon et à Neptune une récompense
          qu'il refusa de payer une fois la ville construite et manqua de récompenser Hercule de la somme promise pour la délivrance de sa fille Hésione.
\end{enumerate}




\section{Livre 2 - Les arbres et la vigne}


\begin{enumerate}
      \item Qu'est-ce que le \textcolor{red}{moût} ? : Jus de raisin non fermenté.
      \item Qu'est-ce que le \textcolor{red}{saulaie} ? :
            \img{0.5}{saulaie.jpg}{Saulaie}{53}
            \item Qu'est-ce que les \textcolor{red}{frondaisons} ? : Feuillage ou époque du feuillage des arbres.
            \item Qu'est-ce que \textcolor{red}{l'orme} ? :
            \img{0.5}{orme.jpg}{Orme}{55}
            \item Qu'est-ce qu'un \textcolor{red}{scion} ? : Scion est un terme tiré du vocabulaire de l’horticulture et
            qui désigne une branche souple fraichement poussée.
            \item Qu'est-ce qu'une \textcolor{red}{bouture} ?  : Fragment d'un végétal mis en terre pour faire une nouvelle pousse.
            \item Où est \textcolor{red}{l'Ismare} ? :
            \img{0.3}{Ismare.png}{Mont Ismare}{56}
      \item Où est le \textcolor{red}{Taburne} ? :
            \img{0.3}{Taburne.png}{Mont Taburne}{57}
            \item Qu'est-ce que les \textcolor{red}{provins} ? : Rejet de vigne qui se développe et prend racine en terre à partir de la plante mère.
            \item Qu'est-ce que le \textcolor{red}{surgeon} ? : Rejeton qui pousse au pied d'un arbre.
            \item Quel est \textcolor{red}{l'arbre ombreux dont Hercule se tressa une couronne} ? : Il s'agit du peuplier blanc. Pluton, ayant aimé l'Océanide
            Leucé, fit naître après sa mort sur les bords de l'Achéron un arbre blanc appelé leucé, des feuilles duquel Hercule se couronna
            à sa sortie des Enfers.
      \item Qui est le père \textcolor{red}{Chaonien} ? : Il s'agit de l'Oracle de Jupiter à Dodone en Chaonie.
      \item Qu'est-ce que \textcolor{red}{enter} ? : Greffer / Assembler.
      \item Qu'est-ce que \textcolor{red}{l'orne} ? :
            \img{0.5}{orne.jpg}{Orne}{61}
            \item Qu'est-ce qu'une \textcolor{red}{tunique} (arbre) ? : Enveloppe qui protège certains végétaux.
            \item Qu'est-ce que le \textcolor{red}{liber} ? : Tissu végétal situé entre l'écorce et le bois d'un arbre.
            \item Pourquoi le \textcolor{red}{liber humide} ? : Le liber est humide du suc nommé cambium, situé entre l'écorce et l'aubier.
            \item Qu'est-ce que \textcolor{red}{l'aubier} ? : L'aubier est la partie de l’arbre se situant entre le bois de cœur ou duramen et le cambium.
            \item Qu'est-ce que \textcolor{red}{l'orchade} ? : Il s'agit d'un olivier au fruit ovale et tendre et assez huileux.
            \item Qu'est-ce que \textcolor{red}{le verge} ? : Il s'agit d'un olivier au fruit très allongé.
            \item Qu'est-ce que \textcolor{red}{la pausie} ? : Il s'agit d'un olivier au fruit très charnu et très huileux.
            \item Qui est \textcolor{red}{Alcinoüs} ? : Il s'agit du roi de Phéacie célèbre dans l'Odyssée.
            \item Où est \textcolor{red}{Lesbos} ? :
            \img{0.4}{Lesbos.png}{Lesbos}{65}
            \item Qu'est-ce que \textcolor{red}{le sarment} ? : Rameau de vigne.
            \item Où est le \textcolor{red}{Méthymne} ? :
                  \img{0.4}{Methymne.png}{Méthymne}{66}
            \item Où est \textcolor{red}{Thasos} ? :
                  \img{0.4}{Thasos.png}{Thasos}{67}
            \item Où est \textcolor{red}{Maréotis} ? :
                  \img{0.4}{Mareotis.png}{Maréotis}{68}
            \item Qu'est-ce que le \textcolor{red}{Psithie} ? : Il s'agit du vin grec.
            \item Qu'est-ce que le \textcolor{red}{Lagéos} ? : Il s'agit d'un vin roux, couleur de lièvre, produit en Crète.
            \item Qu'est-ce que le \textcolor{red}{Rhétique} ? : Il s'agit d'un vin renommé aux environs de Vérone. Auguste l'appréciait beaucoup.
            \item Qu'est-ce que le \textcolor{red}{Falerne} ? : Il s'agit du vin le plus renommé (avec le Cécube) des vins latins; on le récoltait au pied
                  du mont Massique, en Campanie.
            \item Où est \textcolor{red}{Aminée} ? : Pas loin de Naples.
            \item Qu'est-ce que le \textcolor{red}{Phanée} ? : Vignoble d'un promontoire de l'île de Chio.
            \item Qu'est-ce que \textcolor{red}{l'aulne} ? :
            \img{0.5}{Aulne.jpg}{Aulne}{70}
            \item Qui sont les \textcolor{red}{Gélons} ? : Il s'agit d'un peuple de scythe, à l'ouest du Borysthène (le Dnieper actuel); ils étaient tatoués,
            comme beaucoup de peuples barbares, qui se peignaient le corps pour effrayer leurs ennemis.
            \item Qu'est-ce que \textcolor{red}{l'ébénier} ? :
            \img{0.5}{Ebenier.jpg}{Ebénier}{72}
      \item Qu'est-ce que \textcolor{red}{l'acanthe} ? :
            \img{0.4}{Acanthe.jpg}{Acanthe}{73}
            \item Où est la \textcolor{red}{Médie} ? :
            \img{0.4}{Medie.png}{Royaume de Médie}{74}
      \item Qu'est-ce que la \textcolor{red}{pomme salutaire du Médie} ? : Il s'agit du citron.
      \item Qu'est-ce que l'\textcolor{red}{Hermus} ? : Dieu-fleuve associé au Gediz, dans la mythologie grecque.
            \img{1}{Gediz.png}{Gediz}{76}
            \item Qu'est-ce que le \textcolor{red}{taureau soufflant du feu} ? : Il s'agit d'une allusion aux légendes de Jason : soumis par le roi de Colchide à
            une épreuve pour obtenir la Toison d'or, Jason se vit forcé d'atteler à une charrue deux taureaux soufflant du feu par leurs naseaux.
      \item Qu'est-ce que le \textcolor{red}{Massique} ? : Le Massique est un vin de qualité produit en Campanie dans l'Antiquité.
      \item Où est le \textcolor{red}{Clitumne} ? : Le Clitumne (latin Clitumnus ; italien Clitunno) est une rivière italienne, qui traverse
            l'Ombrie, dans la province de Pérouse, et se jette dans le Topino, affluent du Tibre.
      \item Quel est la \textcolor{red}{croyance ancienne à propos du Clitumne} ? : Les anciens croyaient que ses eaux cristallines blanchissaient
            le poil des animaux qui s'y baignaient.
      \item Qu'est-ce que les \textcolor{red}{aconits} ? :
            \img{0.5}{Aconit.jpg}{Aconit}{78}
            \item Où est le \textcolor{red}{Larius} ? :
            \img{0.3}{Come.png}{Lac de Côme ou Larius}{80}
      \item Où est le \textcolor{red}{Benacus} ? :
            \img{0.3}{Garde.png}{Lac de Garde ou Benacus}{81}
            \item Où est le \textcolor{red}{Lucrin} ? :
            \img{0.3}{Lucrin.png}{Lac de Lucrin}{82}
      \item Où est l'\textcolor{red}{Averne} ? :
            \img{0.3}{Averne.png}{Lac d'Averne}{83}
      \item Qu'est-ce que les \textcolor{red}{Marses} ? : Peuple belliqueux établi autour du lac Fucin (aujourd'hui lac asséché de Celano).
            \img{0.3}{Celano.png}{Lac Fucin (ou asséché de Celano)}{84}
      \item Qu'est-ce que les \textcolor{red}{Sabelliens} ? : Les Sabelliens, peuple de la Sabine auquel appartenaient les Marses susnommés.
      \item Qu'est-ce que le \textcolor{red}{Ligure} ? : Habitant du golfe actuel de Gênes.
            \img{0.3}{Gênes.png}{Golfe de Gênes}{85}
            \item Qu'est-ce que les \textcolor{red}{Volsques} ? : Il s'agit d'un peuple de Latium.
            \img{0.3}{Latium.png}{Latium}{86}
      \item Qu'est-ce que les \textcolor{red}{Décius} ? : Il s'agit d'une dynastie d'empereurs romains.
      \item Qu'est-ce que les \textcolor{red}{Marius} ? : Il s'agit du vainqueur de Jugurtha, et des Cimbres et des Teutons.
      \item Qu'est-ce que les \textcolor{red}{grands Camille} ? : Camille, le vainqueur des Gaulois à l'Allia.
      \item Qu'est-ce que les \textcolor{red}{Scipions} ? : Les deux Africains, Cornélius Scipion, le vainqueur d'Annibal à Zama, et Scipion Emilien,
            le destructeur de Carthage.
            \item Que veut dire \textcolor{red}{rouvrir les fontaines sacrées} ? : Cela signifie entreprendre de nouveau à écrire de la poésie pastorale.
            \item Qu'est-ce que le \textcolor{red}{poème d'Ascra} ? : Il s'agit du poème d'Hésiode, né à Ascra, en Béotie, intitulé Les Travaux et les Jours.
            \item Qu'est-ce qu'\textcolor{red}{une patère} ? : Sorte de coupe ancienne pour boire.
            \item Qu'est-ce que \textcolor{red}{Tarente} ? : Il s'agit d'un port de l'Italie méridionale.
            \img{0.3}{Tarente.png}{Port de Tarente}{87}
            \item Où est le \textcolor{red}{Mincio} ? :
            \img{0.4}{Mincio.png}{Cours d'eau du Mincio}{88}
            \item Qu'est-ce que le \textcolor{red}{serpolet} ? :
            \img{0.5}{Serpolet.jpg}{Thym Serpolet}{89}
      \item Qu'est-ce que le \textcolor{red}{romarin} ? :
            \img{0.5}{Romarin.jpg}{Romarin}{90}
            \item Où est \textcolor{red}{Capoue} ? :
            \img{0.5}{Capoue.png}{Capoue}{91}
      \item Où est le \textcolor{red}{Clain} ? :
            \img{0.4}{Clain.png}{Cours d'eau du Clain}{92}
      \item Où est \textcolor{red}{Acerre} ? :
            \img{0.4}{Acerre.png}{Acerre}{93}
      \item Quelle est \textcolor{red}{l'histoire des habitants d'Acerre} ? : Le Clain débordant fréquemment et dangereusement, les habitants d'Acerre,
            qu'il traversait, durent abandonner leur ville.
            \item Qu'est-ce que les \textcolor{red}{déblais} ? : Gravats, débris ou terre que l'on retire d'un terrain.
            \item Qu'est-ce que la \textcolor{red}{poix} ? : Matière molle et visqueuse obtenue avec des résines et des goudrons végétaux.
            \item Qu'est-ce que le \textcolor{red}{vigneron} ? : Personne qui exploite la vigne et élabore le vin.
            \item Qu'est-ce qu'un \textcolor{red}{arpent} ? : Surface peu étendue de terrain.
            \item \textcolor{red}{Inspiration de La Fontaine} ? : Ici, "le chêne vert surtout, dont la tête s'élève autant vers les brises éthérées
            que sa racine s'enfonce vers le Tartare." La Fontaine, Le Chêne et le Roseau : "Celui de qui la tête au ciel était voisine/
            Et dont les pieds touchaient à l'empire des morts."
            \item Quel est \textcolor{red}{l'oiseau blanc qui arrive au printemps vermeil} ? : Il s'agit de la cigogne.
            \item Qu'est-ce qu'un \textcolor{red}{tesson} ? : Débris de verre ou de poterie, morceau qui s'est créé suite à un bris.
            \item Qu'est-ce qu'un \textcolor{red}{bident} ? : Hoyau à deux dents, d'environ trois pieds.
            \item Qu'est-ce que \textcolor{red}{récalcitrant} ? : Qui résiste avec entêtement.
            \item Qu'est-ce que \textcolor{red}{l'exubérance} ? : Abondance, développement excessif, trop grande richesse. Synonyme : luxuriance.
            \item Qu'est-ce qu'un \textcolor{red}{outrage} ? : Préjudice physique ou moral qui porte atteinte à un individu, à un idéal ou à une opinion.
            \item Comment était la \textcolor{red}{comédie à ses débuts} ? : La comédie est à ses débuts une promenade de gens ivres à travers les cantons, "comés", d'un pays.
            \item Qu'est-ce qu'une \textcolor{red}{rasade} ? : Verre d'alcool, ou de n'importe quel autre liquide, qui a pour caractéristique d'être rempli jusqu'aux bords.
            \item Qu'est-ce qu'une \textcolor{red}{outre} ? : Sac cousu à partir de peau de bête et utilisé comme récipient d'eau.
            \item Quel est le \textcolor{red}{jeu qui consistait à sauter dans les molles prairies par-dessus des outres huilées} ? : Ce jeu accompagnait les fêtes de
                  Dionysos et consistait à sauter à cloche-pied sur des outres huilées et à s'y maintenir en équilibre. Le gagnant recevait en prix une outre pleine de vin.
            \item Qui sont les \textcolor{red}{Ausoniens} ? : Peuple de la côte occidentale de l'ancienne Italie : les poètes donnent souvent le nom
                  d'Ausonie à toute l'Italie.
            \item Qu'est-ce que \textcolor{red}{fescennin} ? : Qui appartient à un genre de poésie satirique, généralement grossière et licencieuse.
            \item Qu'est-ce que le \textcolor{red}{rebut} ? : Ce qu'il y a de mauvais, ce qu'on jette après un tri, ce qui est destiné à aller à la poubelle.
            \item Qu'est-ce que le \textcolor{red}{houx} ? :
                  \img{0.5}{Houx.jpg}{Houx}{94}
                  \item Qu'est-ce qu'un \textcolor{red}{fourré} ? : Endroit épais et touffu d'un bois.
                  \item Qu'est-ce qu'un \textcolor{red}{cytise} ? : Arbuste aux fleurs en grappes jaunes. Cette plante appartient à la famille des légumineuses.
                        \img{0.5}{Cytise.jpg}{Cytise}{95}
                  \item Où est le \textcolor{red}{Cytore} ? :
                        \img{0.3}{Cytore.png}{Cytore}{96}
                        \item Où est \textcolor{red}{Naryce} ? : 
                        \img{0.3}{Naryce.png}{Naryce}{98}
                                                                              
\end{enumerate}





\end{document}