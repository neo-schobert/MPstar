\documentclass[a4paper, 11pt, hidelinks]{article}
\usepackage{bookmark}
\usepackage[utf8]{inputenc} 
\usepackage[T1]{fontenc}
\usepackage{lmodern}
\usepackage{graphicx}
\usepackage[french]{babel}
\usepackage{geometry}
\usepackage{eucal}
\usepackage{caption}
\usepackage{float}
\usepackage{url}
\usepackage{amsmath}
\usepackage{amssymb}
\usepackage{color}
\usepackage{hyperref}
\usepackage{cancel}
\usepackage{tikz}
\usepackage{mathrsfs}  
\usepackage{esvect}
\usepackage[standard]{ntheorem}
\usepackage{romanbar}
\usepackage{titlesec}



\geometry{hmargin=2cm,vmargin=1.5cm}

\tikzset{
  treenode/.style = {shape=rectangle, rounded corners,
                     draw, align=center,
                     top color=white, bottom color=blue!5},
  root/.style     = {treenode, font=\Large, bottom color=red!10},
  env/.style      = {treenode, font=\ttfamily\normalsize},
  dummy/.style    = {circle,draw}
}

\newcommand{\prp}{\large \textbf{Proposition :} \large}

\newcommand{\tm}{\large \textbf{Théoreme :} \large}

\newcommand{\ex}{\textcolor{green}{Exemple :} }

\newcommand{\dm}{\textcolor{red}{\textbf{Démo :} } }

\newcommand{\de}{\large \textbf{Définition} \large }

\newcommand{\rmq}{\textbf{Remarque :} }

\newcommand{\bs}{\bigskip}

\newcommand{\voca}{\textcolor{blue}{\textbf{Vocabulaire} } }

\newcommand{\lem}{\textcolor{red}{\textbf{Lemme :} } }

\newcommand{\rb}[1]{\Romanbar{#1}}

\newcommand{\cit}{\textcolor{blue}{\textbf{Citation : }}}

\newcommand{\trinom}[3]{\begin{pmatrix}
    #1 \\
    #2 \\
    #3
\end{pmatrix}}

\newcommand{\quadrinom}[4]{\begin{pmatrix}
    #1 \\
    #2 \\
    #3 \\
    #4 \\
\end{pmatrix}}

\newcommand{\pentanom}[5]{\begin{pmatrix}
    #1 \\
    #2 \\
    #3 \\
    #4 \\
    #5
\end{pmatrix}}

\newcommand{\hexanom}[6]{\begin{pmatrix}
    #1 \\
    #2 \\
    #3 \\
    #4 \\
    #5 \\
    #6 
\end{pmatrix}}

\newcommand{\serie}[2]{\displaystyle\sum_{#1 =0}^{+\infty} #2_{#1} }

\newcommand{\tend}{\underset{n \to + \infty}{\longrightarrow} }

\newcommand{\Lra}{\Leftrightarrow}

\newcommand{\lra}{\leftrightarrow}

\newcommand{\Ra}{\Rightarrow}

\newcommand{\ra}{\rightarrow}

\newcommand{\la}{\leftarrow}

\newcommand{\La}{\Leftarrow}

\newcommand{\dsum}[2]{\displaystyle\sum_{#1}^{#2} }

\newcommand{\dint}[2]{\displaystyle\int_{#1}^{#2} }

\newcommand{\ntend}{\underset{n \to + \infty}{\not \longrightarrow} }

\newenvironment{lmatrix}{$ \left|\begin{array}{l} }{\end{array}\right.$}

\newcommand{\img}[4]{\begin{figure}[!ht]
    \centering
    \includegraphics[scale=#1 ]{#2}
    \caption{#3}
    \label{#4}
    \end{figure} }    
\begin{document}

\newcommand{\grad}[1]{\vv{grad}#1}


\title{Résumé}
\author{Schobert Néo}

\maketitle

\tableofcontents



\newpage


\section{Virgile - Les Géorgiques} 


\subsection{Livre 1}

\begin{itemize}
    \item PREAMBULE \begin{itemize}
        \item Dédicace à Mécène; sujet de chacun des livres (p $37$)
        \item Invocation aux dieux champêtres (p $37$)
        \item Invocation à Auguste qui prendra place dans le ciel (p $39$)
    \end{itemize}
    \item LE TRAVAIL DES CHAMPS \begin{itemize}
        \item Au retour du printemps, il faut labourer la terre; les quatre labours annuels (p $41$)
        \item Chaque terre ayant ses qualités propres, on réglera d'après la nature du fonds le temps et le nombre des façons à lui donner (p $41$)
        \item Les méthodes de culture : la jachère; les assolements; la fumure; l'incinération des éteules sur place; le hersage et les labours qui se recoupent (p $42$)
        \item Après les semailles, conditions et travaux favorables aux céréales (p $44$)
        \item Mais à ce travail font obstacle les ennemis du laboureur; car Jupiter a imposé aux mortels la loi du progrès laborieux; d'où la nécessité de lutter contre la rouille, les plantes parasites, les oiseaux et l'ombre (p $45$)
        \item Les armes du paysan; fabrication de la charrue (p $47$)
        \item Autres préceptes; établissement de l'aire; présages procurés par la floraison de l'amandier; choix et traitement des semences pour éviter qu'elle ne dégénèrent (p $49$)
    \end{itemize}
    \item METEOROLOGIE \begin{itemize}
        \item Il faut observer les astres qui indiquent le moment de semer les différentes graines (p $50$)
        \item Description des zones célestes et des constellations qui indiquent le temps propice à chaque travail (p $52$)
        \item Occupation pour les jours de pluie et les jours de fêtes (p $54$)
        \item La lune a marqué dans le mois les jours favorables ou défavorables  (p $54$)
        \item Travaux à exécuter de nuit, à l'aurore, à la veillée, en plein été, en hiver (p $55$)
        \item Les méfaits causés par les orages violents imposent la vigilance et l'observation des astres (p $57$)
        \item Avant tout, il faut honorer les dieux et en particulier Cérès  (p $58$)
        \item En outre Jupiter a fixé les signes qui permettent de prévoir le temps (p $59$)
        \item Pronostics de mauvais temps fournis par les éléments, le tonnerre, les oiseaux, la flamme de la lampe (p $59$)
        \item Pronostics de beau temps fournis par les astres, les oiseaux (p $61$)
        \item Pronostics lunaires (p $63$)
        \item Pronostics solaires (p $64$)
    \end{itemize}
    \item FINALE \begin{itemize}
        \item C'est le soleil qui annonça la guerre civile et tous les maux qui ont suivi la mort de César (p $65$)
        \item Prière aux dieux de la patrie, pour que le jeune prince puisse ramener la paix dans le monde et la prospérité dans les campagnes délaissées (p $67$)
    \end{itemize}
\end{itemize}



\subsection{Livre 2}

\begin{itemize}
    \item PREAMBULE \begin{itemize}
        \item Invocation à Bacchus, dieu de la vigne et des arbres (p $73$)
        \item Les arbres naissent spontanément ou se reproduisent de diverses manières, que les cultivateurs doivent apprendre (p $73$)
        \item Que Mécène daigne participer à la tâche du poète (p $75$)
    \end{itemize}
    \item PRECEPTES GENERAUX D'ARBORICULTURE \begin{itemize}
        \item Amélioration des espèces par procédés appropriés, et en particulier par la greffe (p $75$)
        \item Du choix des espèces (p $77$)
        \item Les productions varient avec le terrain et le climat (p $79$)
        \item Mais aucune région ne peut rivaliser avec l'Italie (p $81$)
        \item À chaque terrain convient un genre de culture (p $84$)
        \item Moyens de reconnaître la nature du sol (p $87$)
    \end{itemize}
    \item CULTURE DE LA VIGNE \begin{itemize}
        \item Plantation du vignoble; défonçage, transplantation, disposition des plants, profondeur des fosses (p $89$)
        \item Autres précautions à prendre : éviter de planter des oliviers entre les vignes, de planter la vigne quand il gèle; le printemps est pour ce travail la meilleure saison (p $91$)
        \item Hymne au printemps (p $92$)
        \item Soins exigés par les jeunes plants; la taille (p $93$)
        \item Protection des vignes contre les bêtes nuisibles et surtout contre le bouc, que l'on immole à Bacchus (p $94$)
        \item La vigne même adulte réclame des soins continus (p $96$)
    \end{itemize}
    \item CULTURE DES OLIVIERS ET DES AUTRES ARBRES \begin{itemize}
        \item Moins pénible que la viticulture est la culture de l'olivier, des arbres fruitiers, et des essences forestières, qui sont si utiles à l'homme (p $97$)
    \end{itemize}
    \item FINALE : ELOGE DE LA VIE CHAMPÊTRE \begin{itemize}
        \item Bonheur des paysans (p $99$)
        \item Le poète aspire à vivre à la campagne (p $101$)
        \item Calme et pureté de la vie rurale (p $102$)
    \end{itemize}
\end{itemize}



\subsection{Livre 3}

\begin{itemize}
    \item PREAMBULE \begin{itemize}
        \item Invocation à Palès et aux divinités des troupeaux (p $109$)
        \item En traitant ce sujet, le poète espère triompher; vainqueur, il élèvera un temple et célébrera des jeux à la gloire du prince (p $110$)
        \item Appel à Mécène (p $113$)
    \end{itemize}
    \item LE GROS BETAIL \begin{itemize} 
        \item Choix des génisses destinées à la reproduction (p $114$)
        \item Choix des étalons (p $115$)
        \item Soins à donner aux mâles et aux femelles avant l'accouplement (p $118$)
        \item Soins à donner aux femelles pleines; il faut les mettre à l'abri des taons (p $119$)
        \item Soins à donner aux nouveau-nés (p $120$)
        \item Dressage des veaux destinés aux charrois (p $120$)
        \item Dressage des poulains (p $121$)
        \item Nécessité de les soustraire à l'amour, qui mine les taureaux et les rend furieux (p $122$)
        \item Puissance de l'amour, maître de la création (p $124$)
        \item En particulier, il affolle les juments (p $125$)
    \end{itemize}
    \item LE PETIT BETAIL \begin{itemize} 
        \item Abordant ce sujet difficile, le poète invoque de nouveau Palès (p $127$)
        \item Les étables des brebis et des chèvres; l'élevage des chèvres est productif et facile; soins à donner au troupeau, en hiver et en été  (p $127$)
        \item Bergers nomades en Libye (p $130$)
        \item Par opposition, vie casanière des Scythes pendant la nuit hivernale (p $130$)
        \item La laine (p $132$)
        \item Le lait et le fromage (p $133$)
        \item Les chiens de garde et de chasse (p $133$)
        \item La lutte contre les serpents (p $134$)
        \item La lutte contre les maladies des ovins (p $135$)
    \end{itemize}
    \item EPILOGUE \begin{itemize} 
        \item Tableau de l'épizootie qui a ravagé le Norique et les bords du Timave (p $137$)
    \end{itemize}
\end{itemize}



\subsection{Livre 4}


\begin{itemize}
    \item PREMIERE DIRECTIVES \begin{itemize} 
        \item Nouvelle invocation à Mécène, au moment de chanter le miel et la vie des abeilles (p $145$)
        \item Situation des ruches (p $145$)
        \item Conditions qu'elles doivent remplir (p $147$) 
        \item Ce que doit faire l'apiculteur lorsque les abeilles sortent pour butiner, pour essaimer ou pour se battre (p $148$)
        \item Choix du roi; les deux espèces d'abeilles (p $150$)
        \item Il faut retenir les abeilles dans un jardin fleuri  (p $150$)
        \item Ce serait le moment, si le temps ne pressait, de parler des jardins tels que celui du vieillard de Tarente (p $151$)
    \end{itemize}
    \item LA CITE DES ABEILLES \begin{itemize} 
        \item Organisation et division du travail (p $153$)
        \item Sacrifice de l'individu à la communauté : propagation de l'espèce, risques courus dans l'intérêt général, obéissance au roi (p $156$)
        \item Ces mœurs ont fait penser que les abeilles participaient de l'âme divine, qui anime tous les êtres (p $157$)
    \end{itemize}
    \item PRESCRIPTIONS DIVERSES \begin{itemize} 
        \item Récolte du miel au printemps et à l'automne (p $158$)
        \item Autres soins pour encourager les abeilles (p $159$)
        \item Comment reconnaître et soigner leurs maladies (p $159$)
        \item Si l'espèce vient à disparaître, l'apiculteur aura recours au moyen employé par Aristée; il laissera se putréfier le cadavre d'un veau, d'où sortira un nouvel essaim (p $161$)
    \end{itemize}
    \item EPISODE D'ARISTEE \begin{itemize} 
        \item Ayant perdu ses abeilles, Aristée demanda la cause et le remède à sa mère Cyrène (p $163$)
        \item Celle-ci lui conseilla de consulter Protée (p $167$)
        \item Avec l'assistance de sa mère, Aristée réussit à saisir le dieu et à le faire parler (p $168$)
        \item Révélation de Protée; Aristée a causé sans le vouloir la mort d'Eurydice; Orphée, son époux, est descendu aux enfers et l'a ramenée; mais, oubliant la condition imposée, il s'est retourné vers elle; Eurydice aussitôt s'est évanouie dans les ténèbres infernales; Orphée inconsolable a péri, déchiré par les femmes qu'il méprisait (p $170$)
        \item Après la disparition de Protée, Cyrène complète ces révélations en indiquant à son fils les sacrifices expiatoires et en lui recommandant d'abandonner le corps des victimes (p $175$)
        \item Aristée obéit et voit grouiller hors des cadavres un nouvel essaim (p $176$)
    \end{itemize}
    \item EPILOGUE \begin{itemize} 
        \item Ainsi chantait Virgile à Naples (p $177$)
    \end{itemize}
\end{itemize}







\end{document}